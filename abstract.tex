%
% File hicss51.tex
%
% Contact: Holm Smidt, hsmidt@hawaii.edu
%%
%%
%% Based on the style files for ACL 2015 by 
%% car@ir.hit.edu.cn, gdzhou@suda.edu.cn


\documentclass[10pt]{article}
\input{hicss51-packages.tex}
\usepackage{blindtext}
\newcommand{\sansserifformat}[1]{\fontfamily{cmss}{ #1}}%

%\setlength\titlebox{5cm}

% You can expand the titlebox if you need extra space
% to show all the authors. Please do not make the titlebox
% smaller than 5cm (the original size).


\title{Improving Waste Collection Procedures \\ In Practical Smart City Implementations}

\author{
  Jan D{\"u}nnweber  \\
  Ostbayerische Technische \\ Hochschule Regensburg \\
  {\underline{ jan.duennweber@othr.de} }\\\And 
  Amitrajit Sarkar \\
 Ara Institute \\ of Canterbury \\
  {\underline{amit.sarkar@ara.ac.nz}} \\\And
  Vimal Kumar Puthiyadath \\
 KPIT Technologies Ltd \\
  {\underline{Vimal.Puthiyadath@kpit.com}} 
  }

\date{}

\begin{document}
\maketitle
\begin{abstract}
Computer-based Improvements to waste collection procedures are often a part of smart city 
initiatives. When we envision an ideal waste collection vehicle, it will arrive at every
container exactly at the time when it is fully loaded. Beyond doubt, this will 
reduce traffic and support environmentally friendly intentions like an expansion 
of waste separation as it will make more containers manageable. 
An obvious difficulty of putting that vision into practice is that  
collection vehicles cannot always be where they are needed. 
Knowing the best time for emptying a container is insufficient for 
finding the optimal collection route.
Therefore, we compare three different approaches to reducing the waste collection
times by the use of networked fill-level sensors: Regensburg,
Christchurch and Pune. Our analysis shows that the most efficient collection schedules 
result from adapting field-tested routes frequently on the basis of 
current sensor measurements and shortcuts resulting from route 
optimization computations.
\end{abstract}

\section{Introduction}
Even with the latest IoT technology like networked sensors and simulations
forecasting the collection times of megacities, practical implementations of 
on-demand waste collection still have difficulties in keeping up with the 
prognosticated improvements.
A collection vehicle that drives obstinately from the most heavily filled 
container to one with the fill-grade closest to that will obviously need 
more time in the majority of cases than a vehicle following a fixed plan, 
since heavily filled containers are probably positioned far apart from 
each other. Finding a smarter route leads us to the classic 
{\it vehicle routing problem} (VRP~\cite{Dantzig59}), an instance of the 
{\it Travelling Salesman Problem (TSP)} with the added constraint that we 
need to return to the starting point after visiting a fixed number of points, 
since the collection vehicle has a limited capacity. 
Thus, we don't need to find the minimum Hamiltonian
circle through all the points but multiple circles forming some kind of
clover leaf. However, finding the best route to collect the waste 
containers does not only require to consider the distances between 
the single containers. Containers which are only filled to a certain level should
be skipped, i.\,e. we are dealing with a instance of a dynamic route 
planning problem, which is also the subject of more recent 
research~\cite{Chen16}.

There are $\frac{n!}{2}$ different routes connecting $n$ containers. 
For comparing all routes between only $10$ containers, this means 
3628800 routes must be analyzed. Modern waste collection
vehicles can be loaded with $\approx 400$ container of 120 
liters~\cite{hyundai18}.
$400!$ is a $882$-digit number. Taken into account that skipping 
containers with little load, means the vehicle has to pickup an other
one where it usually does not drive to, solving our dynamic VRP requires
to solve a new problem of that size, every time when the fill-level
measurements are updated. Nowadays, supercomputers can deal with 
such problem sizes~\cite{Burkhovetskiy2017}. However, the
presented projects deal with approximate solutions, which
can be found using a standard PC or an on-board computer in
the garbage truck. Therefore, the presented work might be relevant 
for automating the navigation of future self-driving garbage 
trucks, like the one Volvo started testing in Brussels 
recently~\cite{volvo17}.

The approximate solutions presented in this paper are based on 
the {\it ant colony optimization} (ACO~\cite{Dorigo04}). This 
approach has been proven suitable for dynamic VRP instances
in a simulation, where the road network and the related 
traffic were taken into account~\cite{Karadimas2008}. ACO is a {\it swarm intelligence}
procedure, i.\,e. not an individual (a simulated ant in the case of 
ACO) solves a problem but a group. For finding optimal routes,
the simulation starts with letting the ants take random paths
until they reach their destination. This random walk is optimized
iteratively: each ant leaves a pheromone trail behind it which
evaporates after a certain number of iterations. In every iteration
the pheromone intensity of the shorter paths increases because
whenever a simulated ant can choose among multiple paths, it takes
the one with the highest pheromone intensity. This means, in higher
iterations, the paths are no more randomly chosen but influenced 
by the most successful ants from preceding iterations, which are
the ones whose pheromone trails did not evaporate until their 
followers reached them, since they were on the shortest paths.

The rest of this paper is structured as follows: 
Section~\ref{sec:Regensburg} shows how the city of Regensburg benefits from using 
fill-level sensing and ACO-based route optimization for collecting their biological
waste containers. 
Section~\ref{sec:Christchurch} introduces {\it LevelSense} in Christchurch, 
an IoT-based approach to on-demand waste collection, which also uses ACO and is a 
part of a larger {\it Smart City} initiative in New Zealand, the {\it PiP-IOT project}. 
Section~\ref{sec:Pune} introduces another route optimization algorithm, which is used
by {\tt kpit.com} for solving a related problem: Getting all the employees
from various places in Pune to their their offices. 

Section~\ref{sec:concl} looks back on the three projects, which were all 
put into practice, discusses the {\it lessons learned} from these projects,
their benefits and points out some future perspectives.


\section{Regensburg's Smart City Approach: Collecting Biological Waste more Efficiently}
\label{sec:Regensburg}

Our work in Regensburg focuses on improving the collection of biological waste.
Figure~\ref{fig:container} shows a container.

\begin{figure}[h!]
	% Use the relevant command to insert your figure file.
	% For example, with the graphicx package use
    \centering
	\includegraphics[trim={3cm 3cm 3cm 3cm}, clip,width=0.6\linewidth]{biobin}
	% figure caption is below the figure
	\caption{A Biological Waste Container equipped with a fill-level sensor}
	\label{fig:container}       % Give a unique label
\end{figure}

\blindtext

\section{LevelSense and the PiP-IoT project}
\label{sec:Christchurch}

\blindtext

\section{Collecting Employees at {\tt kpit.com}}
\label{sec:Pune}

\blindtext

\section{Conclusion and Future Perspectives}
\label{sec:concl}

\blindtext

% \section{References} 

% List and number all bibliographical references in 9-point Times, single-spaced, at the end of your paper. When referenced in the text, enclose the citation number in square brackets.
% % for example \cite{Jones2015,Smith2015} and \cite{Smith2015}. 
% Where appropriate, include the name(s) of editors of referenced books.

% if added before the last page, this command can help balancing columns
%\addtolength{\textheight}{-.2cm} 

%Bibliography 
\bibliographystyle{ieeetr}
\bibliography{sample}


\end{document}
